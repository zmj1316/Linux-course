\documentclass[notitlepage,cs4size,punct,oneside]{report}
\usepackage{CJK}
\usepackage{graphicx}
\usepackage{url}
\usepackage{xcolor}
\usepackage{url}
\usepackage{listings}

\renewcommand{\ttdefault}{consolas}

\lstset{numbers=left, numberstyle=\tiny, keywordstyle=\color{blue!70}, commentstyle=\color{red!50!green!50!blue!50}, aboveskip=1em, keywordstyle=\bf\color{blue},
   identifierstyle=\bf,escapebegin=\begin{CJK}{UTF8}{hei},escapeend=\end{CJK},extendedchars=false,
   numberstyle=\color[RGB]{0,192,192},
  commentstyle=\color[RGB]{0,96,96},
 stringstyle=\rmfamily\slshape\color[RGB]{245,100,0},language=C}
\begin{document}
\begin{CJK}{UTF8}{hei} % gbsn: 宋体简化字;gkai 楷体简化字;  bsmi 繁体宋书;bkai 繁体楷书


\renewcommand{\abstractname}{摘 \qquad 要}
\renewcommand{\contentsname}{\center 目\qquad\qquad 录}
\renewcommand{\listfigurename}{图 \quad 示 \quad 目 \quad 录}
\renewcommand{\listtablename}{表 \quad 格 \quad 目 \quad 录}
\renewcommand{\appendixname}{附录}
\renewcommand{\chaptername}{章节}
%\renewcommand{\refname}{\center 参 \quad 考 \quad 文 \quad 献}
%\renewcommand{\bibname}{专著}
\renewcommand{\indexname}{\center 索 \qquad 引}
\renewcommand{\figurename}{图}
\renewcommand{\tablename}{表}
%\renewcommand{\pagename}{页}
\bibliographystyle{plain}
\newcommand{\upcite}[1]{\textsuperscript{\cite{#1}}}


\title{作业管理系统需求文档}
\author{***, ***,***}
%\date{年月日}

\maketitle
\tableofcontents
\pagebreak

\chapter{需求分析}
\paragraph{}以文件的形式存储数据,实现三类不同权限用户的操作
\section{总体功能}
\begin{description}
  \item[管理员] 创建、修改、删除、显示(list)教师帐号;教师帐户包括教师工号、教师姓名,教师用户以教师工号登录。\\ \\
 创建、修改、删除课程;绑定(包括添加、删除)课程与教师用户。课程名称以简单的中文或英文命名。
  \item[教师 ] 对某门课程,创建或导入、修改、删除学生帐户,根据学号查找学生帐号;学生帐号的基本信息包括学号和姓名,学生使用学号登录。\\ \\
  布置作业或实验。包括新建、编辑、删除、显示(list)作业或实验等功能。\\ \\
  查找、打印所有学生的完成作业情况。
  \item[学生 ] 在教师添加学生账户后,学生就可以登录系统,并完成作业和实验。\\ \\
  基本功能:编辑作业或实验功能;查询作业或实验完成情况。
\end{description}
\section{详细分析}
\subsection{管理员}
\paragraph{}用户管理功能,可以选择新建用户,修改用户信息,删除用户。修改和是删除通过显示用户列表来进行选择。可以操作所有类型的用户。
\paragraph{}课程管理功能,可以选择新建课程,修改课程信息,删除课程。新建课程同时需要绑定相应的课程教师。
\subsection{教师}
\paragraph{}学生管理功能,可以选择某门课程,查看选课的学生,可以通过学生学号将其添加到该门课程下,或者删除某个学生
\paragraph{}作业管理功能,可以选择某门课程,查看该课程的作业,可以新建、删除、修改作业信息,新建、删除作业的同时会将作业同步到所有学生的信息下。也可以查看某项作业的完成情况。
\subsection{学生}
\paragraph{}提交作业功能,可以查询该学生名下的所有作业,并且选择一项作业完成提交,标记为已完成。

\pagebreak
\end{CJK}
\end{document}
